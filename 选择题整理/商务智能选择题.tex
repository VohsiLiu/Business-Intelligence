\documentclass[10pt,a4paper,UTF8]{ctexart}

\linespread{1.5}
\usepackage{geometry}%用于设置上下左右页边距
	\geometry{left=2.5cm,right=2.5cm,top=3.2cm,bottom=2.7cm}
\usepackage{xeCJK,amsmath,paralist,enumerate,booktabs,multirow,graphicx,subfig,setspace,listings,lastpage,hyperref}
\usepackage{amsthm, amssymb, bm, color, framed, graphicx, hyperref, mathrsfs}
\usepackage{mathrsfs}  
	\setlength{\parindent}{2em}
	\lstset{language=Matlab}%
\usepackage{fancyhdr}
\usepackage{graphicx}
\usepackage{subfloat}
\usepackage{listings}
\usepackage{xcolor}
\usepackage{float}
\usepackage{paralist}
\usepackage{setspace}
\usepackage{titlesec}
\usepackage{enumitem}
\usepackage{hyperref}
\usepackage{multirow}
\usepackage{threeparttable}
\usepackage{tcolorbox}
\usepackage{tabularx}
\usepackage{ulem}
\usepackage{pifont}


\hypersetup{
	colorlinks=true,
	linkcolor=black
}

\setenumerate{partopsep=0pt,topsep=0pt}
\setitemize{itemsep=0pt,partopsep=0pt,topsep=0pt}

\titlespacing*{\section}{0pt}{3pt}{3pt}
\titlespacing*{\subsection}{0pt}{2pt}{2pt}
\titlespacing*{\subsubsection}{0pt}{1pt}{1pt}
\titlespacing*{\paragraph}{0pt}{0pt}{0pt}

\ctexset{secnumdepth=4,tocdepth=4}
\setlength{\parindent}{2pt}
\setstretch{1.5}


\setCJKmainfont[BoldFont={FZHei-B01},ItalicFont={FZKai-Z03}]{FZShuSong-Z01} 
\setCJKsansfont[BoldFont={FZHei-B01}]{FZKai-Z03} 
\setCJKmonofont[BoldFont={FZHei-B01}]{FZFangSong-Z02}
\setCJKfamilyfont{zhsong}{FZShuSong-Z01} 
\setCJKfamilyfont{zhhei}{FZHei-B01} 
\setCJKfamilyfont{zhkai}[BoldFont={FZHei-B01}]{FZKai-Z03} 
\setCJKfamilyfont{zhfs}[BoldFont={FZHei-B01}]{FZFangSong-Z02} 
\renewcommand*{\songti}{\CJKfamily{zhsong}} 
\renewcommand*{\heiti}{\CJKfamily{zhhei}} 
\renewcommand*{\kaishu}{\CJKfamily{zhkai}} 
\renewcommand*{\fangsong}{\CJKfamily{zhfs}}


\definecolor{mKeyword}{RGB}{0,0,255}          % bule
\definecolor{mString}{RGB}{160,32,240}        % purple
\definecolor{mComment}{RGB}{34,139,34}        % green
\definecolor{mNumber}{RGB}{128,128,128} 

\lstdefinestyle {njulisting} {
	basewidth = 0.5 em,
	lineskip = 3 pt,
	basicstyle = \small\ttfamily,
	% keywordstyle = \bfseries,
	commentstyle = \itshape\color{gray}, 
	basicstyle=\small\ttfamily,
	keywordstyle={\color{mKeyword}},     % sets color for keywords
	stringstyle={\color{mString}},       % sets color for strings
	commentstyle={\color{mComment}},     % sets color for comments
	numberstyle=\tiny\color{mNumber},
	numbers = left,
	captionpos = t,
	breaklines = true,
	xleftmargin = 1 em,
	xrightmargin = 0 em,
	frame=tlrb,
	tabsize=4
}

\lstset{
style = njulisting, % 调用上述样式 
flexiblecolumns % 允许调整字符宽度
}


%================= 基本格式预置 ===========================
\usepackage{fancyhdr}
\pagestyle{fancy}
\lhead{\textsc{Business Intelligence}}
\rhead{商务智能}
\cfoot{\thepage}
\renewcommand{\headrulewidth}{0.4pt}
\renewcommand{\theenumi}{(\arabic{enumi})}
\CTEXsetup[format={\bfseries\zihao{-3}}]{section}
\CTEXsetup[format={\bfseries\zihao{4}}]{subsection}
\CTEXsetup[format={\bfseries\zihao{-4}}]{subsubsection}


\renewcommand{\contentsname}{目录}  

%\definecolor{shadecolor}{RGB}{241, 241, 255}
\newcounter{problemname}
\newenvironment{problem}{\stepcounter{problemname}\par\noindent\textbf{\arabic{problemname}.\,}}{\vspace{0.45em}}
\newenvironment{solution}{\noindent \textbf{解析:}\kaishu }{}

\newcommand{\myline}{\uline{\ \ \ \ \ \ \ \ \ \ }}

\begin{document}


\begin{center}
\LARGE\textbf{商务智能选择题整理}
\end{center}

\begin{problem}
关于商务智能,下列论述中正确的有
\myline
\begin{compactenum}[A.]
    \item 商务智能试图以信息化技术自动完成数据到信息、信息到知识的提取过程。
    \item 从信息技术方面来看,商务智能涵盖了数据仓库、多维建模、ETL、OLAP、仪表盘、报表查询、数据统计、数据挖掘等多种相关技术。
    \item 从商业应用方面来看,商务智能不仅支持最新的IT技术,同时也应该提供打包的商务解决方案。
    \item 从层次结构方面来看,可以将商务智能体系结构划分为:数据源层、数据获取层、数据存储和管理层及信息访问/展现层。
    \item 鉴于商务智能的重要作用,企业在进行信息化时,应以企业数据模型为蓝本,同步事务处理系统和分析处理系统。
\end{compactenum}
\end{problem}


\begin{problem}
在进行多维建模时,下列论述中正确的有
\myline
\begin{compactenum}[A.]
    \item 由于规范化引入了查询时的额外开销,在维度表规模不大的前提下,一般不考虑采用雪花模型。然而,在应对大型快变维度时,仍然可以使用采用雪花模型/微型维度的方式节约存储开销。
    \item 非事实型事实表在多个维度间建立连接关系。但是由于没有独立的度量值,非事实型事实表无法独立使用,必须依附于其他事实表参与分析应用。
    \item 在采用累积快照进行多维建模时,事实表中记录了单个生命周期中多个关键环节所产生的信息,并使用多个日期类型的维表对这些关键环节进行标记。
    \item 为方便与操作型数据环境的对接,简化ETL的处理过程,用以连接维度表和事实表的关键字可以直接来源于操作型数据环境的关键字。
    \item 退化维度一般用以对事实表进行分组,并偶尔用以连接操作型数据环境。
\end{compactenum}
\end{problem}


\begin{problem}
关于数据仓库,下列论述中正确的有
\myline
\begin{compactenum}[A.]
    \item 数据仓库是数据库技术进一步发展的必然阶段。由于数据仓库数据容量大,数据模型先进且允许存在冗余,数据仓库正日益替代数据库,成为主流的数据存储技术。
    \item 原子层拥有数据仓库最低粒度的数据,因此,在数据通过ETL进入原子层时,应与数据源保持相同粒度。
    \item 由于数据仓库的“不可更新”特性,数据仓库中的数据实际上是“滞后”的。所以数据仓库必须定期/不定期的采用刷新方式,将数据库等数据源中最新的数据变化反应到数据仓库中来。
    \item 数据仓库是一种反映主题的全局性数据组织,在执行周期性分析应用或局部分析应用时,往往效率不高。在这种情况下,可以按部门或个人分别建立反映各个子主题的局部性数据组织,称作数据集市。
    \item 数据仓库内部以“快照”的数据结构为中心来组织。快照通常包括关键字、时间、非关键字的主要数据和二级数据四个部分,其中非关键字的主要数据是数据仓库用以存放信息的主要部分。
\end{compactenum}
\end{problem}


\begin{problem}
关于OLAP,下列论述中正确的有
\myline
\begin{compactenum}[A.]
    \item 一般来说,ROLAP查询效率优于MOLAP,但装载性能劣于MOLAP。
    \item 由于采用了多维数据库(MDDB),MOLAP可以比ROLAP支持更多的维度。
    \item 在ROLAP中,需要采用多维综合引擎,在多维查询和结果及SQL查询和结果之间进行转化。
    \item 在OLAP系统中,可以融合 MOLAP 和 ROLAP 两种技术。采用关系型数据库存储细节数据,使用多维数据库来存放高层次数据或关系型数据库的查询结果。
    \item 在需要对多维模型进行演化时,ROLAP 相对于 MOLAP 更加灵活便利。
\end{compactenum}
\end{problem}


\begin{problem}
关于数据挖掘,下列论述中正确的有
\myline
\begin{compactenum}[A.]
    \item 数据挖掘属于对数据的归纳。
    \item 对数据挖掘的结果的评价分为主观评价和客观评价。一般来说,最常见的客观评价指标是“支持度(兴趣度)” /“置信度”。
    \item 与分类方法不同,聚类方法不需要给定训练数据集和测试数据集,而是使用数据之问的相似程度/相异程度进行类别划分。因此,聚类方法是一种无指导的学习。
    \item 采用相同的数据挖掘过程、数据挖掘方法、数据挖掘算法及相同的阈值和参数,对相同数据集进行多次数据挖掘,所得到的数据挖掘结果也应该是相同的。
    \item 将$B$是$A$的子女记做$A \rightarrow B$,如存在$N_1 \rightarrow N_2$,$N_2 \rightarrow N_3$,$\cdots$,$N_{k-1} \rightarrow N_k$,则称$N_1$是$N_k$的祖先。如果数据库/数据仓库中存放着所有市民之间的子女关系,那么通过数据挖掘能够获取所有市民之间的祖先关系。
\end{compactenum}
\end{problem}


\begin{problem}
商务智能的目标包括
\myline
\begin{compactenum}[A.]
    \item 为商务活动提供自动化解决方案。
    \item 实现商务领域的人工智能。
    \item 提供商务领域的专家系统。
    \item 进行决策支持。
    \item 改善信息访问方式。
\end{compactenum}
\end{problem}


\begin{problem}
在进行多维建模时,下列论述中正确的有
\myline
\begin{compactenum}[A.]
    \item 依据多级体系划分属性,可以将维度表规范化,以雪花模型替代星型模型,从而节约存储空间。然而,由于规范化引入了查询时的额外开销,在维度表规模不大的前提下,一般不考虑采用雪花模型。
    \item 非事实型事实表在多个维度间建立连接关系。由于没有独立的度量值,非事实型事实表无法独立使用,必须依附于其他事实表参与分析应用。
    \item 一般来说,对事实表进行建模时,事务模型粒度最小,周期快照粒度最大,而累积快照粒度介于两者之间。
    \item 为保持历史一致性,维度表的元组属性发生变化时,需要用额外的行或者列来记录历史信息。因此,在对大型快变维度进行处理时,必须采用微型维度、预设波段等方法,将快变维度转化为渐变维度。
    \item 为方便与操作型数据环境的对接,并简化ETL的处理过程,用以连接维度表和事实表的关键字应当直接来源于操作型数据环境。
\end{compactenum}
\end{problem}


\begin{problem}
根据使用的数据结构不同,OLAP可以分为MOLAP和ROLAP两种主要类型,下列论述中正确的有
\myline
\begin{compactenum}[A.]
    \item 一般来说,MOLAP查询效常优于ROLAP,但装载性能ROLAP优于MOLAP。
    \item 由于采用了多维数据库(MDDB),MOLAP可以比ROLAP支持更多的维度。
    \item 在需要对多维模型的维度数量进行调整的情况下,ROLAP相对于MOLAP更加灵活便利。
    \item 在ROLAP技术中,需要采用多维综合引擎,在多维查询和结果及SQL查询和结果之间进行转化。
    \item OLAP的基本操作包括:切片/切块、旋转、上钻(roll up)/下钻(drill down)。其它操作包括跨钻 (drill across)/钻透(drill through)。人们可以通过这些操作浏览数据立方体,从而获得对分析对象的全面理解。
\end{compactenum}
\end{problem}


\begin{problem}
关于数据仓库,下列论述中正确的有
\myline
\begin{compactenum}[A.]
    \item 数据仓库是数据库技术进一步发展的必然阶段。由于数据仓库数据容量大、数据模型先进且允许存在冗余,数据仓库正日益替代数据库,成为主流的数据存储技术。
    \item 原子层拥有数据仓库最低粒度的数据,因此,在数据通过ETL进入原子层时,应与数据源保持相同粒度。
    \item 出于数据仓库的“不可更新”特性,数据仓库中的数据实际上是“滞后”的。这类问题可以通过使用操作数据存储(ODS)加以改善。
    \item 鉴于大多数的数据仓库系统仍然使用关系型数据库加以实现,在线系统需要访问数据仓库中的信息时,依然可以比照数据库的访问方式,采用SQL等进行直接查询。
    \item 即使使用关系型数据库作为数据仓库的后台存储技术,由于目标、性能和使用对象等的不同,数据仓库仍然拥有自身所特有的数据建模思想、索引及优化方法。
\end{compactenum}
\end{problem}


\begin{problem}
关于数据挖掘,下列论述中正确的有
\myline
\begin{compactenum}[A.]
    \item 数据挖掘属于对数据的演绎。
    \item 对数据挖掘的结果的评价分为主观评价和客观评价。一般来说,最常见的客观评价指标是“支持度(兴趣度)” /“置信度”。
    \item 与分类方法不同,聚类方法不需要给定训练数据集和测试数据集,而是使用数据之问的相似程度/相异程度进行类别划分。因此,聚类方法是一种无指导的学习。
    \item 特征规则挖掘可以被用来作为其他数据挖掘方法的前继步骤,通过在“基本关系表”和“概括关系表”上应用其他数据挖掘方法,可以得到不同概念层次上的数据挖掘结果。
    \item 采用相同的数据挖掘过程、数据挖掘方法、数据挖掘算法及相同的阈值和参数,对相同数据集进行多次数据挖掘,所得到的数据挖掘结果也应该是相同的。
\end{compactenum}
\end{problem}




\end{document}
