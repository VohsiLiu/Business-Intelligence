\begin{spacing}{1.2}
    \centering
    \begin{longtable}{|m{1.2cm}<{\centering}|m{5cm}|m{4.4cm}|m{3.5cm}|}
        \hline
        方法        & \multicolumn{1}{c|}{数据仓库与数据集市的关系}                                                                     & \multicolumn{1}{c|}{优点}                                & \multicolumn{1}{c|}{缺点}                          \\ \hline
        自顶向下的结构   & {\kaishu 构建企业数据仓库}:公共中央数据模型;数据再加工;减少冗余和不一致性;搜集历史的、细节的、全局的数据。{\kaishu 基于企业数据仓库构建数据集市}:选定企业模型下的部门主题;聚集数据;建立集市数据对企业数据仓库的依赖关系 & 建立数据集市能够减轻DW访问负载;各部门可以任意处理数据;数据转换和整合在DW阶段统一完成;具备数据缓冲功能 & 成本高、见效慢、数据集市间不共享资源                               \\ \hline
        自底向上的结构   & {\kaishu 构建数据集市}:划定主题区域;快速实施本地自治;易于复制;数据再加工;允许一定的冗余和不一致。{\kaishu 基于数据集市构建企业数据仓库}:确定各数据集市的可用性;模型的合并;消除不同数据集市之间的数据不一致性      & 见效快、启动资金少                                              & 各个部门都要进行数据清理整合;可能造成“蜘蛛网”、数据不一致等问题;总体上没有节约资金      \\ \hline
        总线结构的数据集市 & 不建立数据仓库而直接建立数据集市;各个数据集市不是孤立的,相互之间通过一种共享维表和事实表的“总线结构”紧密联系在一起                                           & 共享维表和事实表,解决了建立数据集市的许多问题                                & 这种结构基于多维模型,应用限制于OLAP;多个数据源直接影响多个集市,造成数据仓库结构不十分稳定 \\ \hline
        企业级数据集市结构 & 在没有数据仓库的前提下用总线结构构建数据集市                                                                                & 数据集市遵循总线结构进行独立开发部署                                     & ETL过程不存在数据结构问题                                   \\ \hline
    \end{longtable}
\end{spacing}
\vspace{-0.5em}