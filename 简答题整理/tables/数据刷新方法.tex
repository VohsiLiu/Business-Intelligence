\begin{spacing}{1.2}
    \centering
    \begin{longtable}{|m{1.2cm}<{\centering}|m{4.7cm}|m{4.7cm}|m{3.3cm}|}
        \hline
        方法      & \multicolumn{1}{c|}{适用情况/实现方法}                                                               & \multicolumn{1}{c|}{优点}                                                         & \multicolumn{1}{c|}{缺点}                             \\ \hline
        时间戳     & 若数据库中的记录有时间属性,则可根据OLTP数据库中的数据有无更新,以及在执行更新操作时数据的修改时间标志来实现数据仓库中数据的动态刷新。   &                                                            & 大多数数据库系统中的数据并不含有时间属性。          \\ \hline
        DELTA文件 & 有些基于OLTP数据库的操作型应用程序在工作过程中会形成一些DELTA文件以记录该应用所作的数据修改操作,可根据该DELTA文件进行数据刷新。 & 采用此方法可避免对整个数据库的对比扫描,具有较高的刷新效率。                             & 这样的应用程序并不普遍,修改现有的应用程序的工作量又太大。  \\ \hline
        建立映象文件  & 在上一次数据刷新后对数据库作一次快照;在本次刷新之前再对数据库作一次快照;比较两个快照的不同,从而确定数据仓库的数据刷新操作。         & 对于数据库和操作型应用无特别要求。                                          & 需要占用大量的系统资源;可能较大地影响原有数据库系统的性能。 \\ \hline
        日志文件    & 一般来说,现代OLTP数据库都有日志文件,可根据OLTP数据库的日志信息来实现数据仓库的数据刷新。                       & 日志是OLTP数据库的固有机制;不会影响原有OLTP数据库的性能;具有比DELTA文件和建立映象文件更高的刷新效率。 & 无法应用于无日志文件机制的遗留数据库系统等。         \\ \hline
    \end{longtable}
	\end{spacing}
\vspace{-0.5em}